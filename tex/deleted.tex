%Follow up work on the project was done to find common misconceptions about cybersecurity. This involved doing 26 think-aloud interviews with students who had taken at least one course in cybersecurity. These students were presented multiple scenarios to which they were then asked a series of interview questions. These interviews yielded some important themes to students misconceptions. Theme 1 was student tend to overgeneralize for example attributing more features to encryption than can be reasonably expected. Theme 2 conflating two concepts together for example hashing and encryption. Theme 3 biases including user bias, physical bias, and personal bias. User bias pertains to their own personal concerns leading them to overlook more relevant concerns. Physical bias meaning students tend to lead to physical solutions to security. Personal bias about countries or cultures that informed incorrect assumptions or beliefs. Theme 4 incorrect assumptions meaning students tend to make assumptions without having a solid basis for making these assumptions \cite{scenarios}. This provided some of the fundamental scenarios the questions were based on and provided guidance in the topics of the questions and some incorrect answers that could evaluate these misconceptions.

%After the initial development of the \gls{cci} and the beginning of the development of the \gls{cca}, the group held a "hackathon" with experts in the field. This hackathon was held in February 2018, and had 17 experts with 13 from academic universities, 2 from industry, and 2 from government. These participants were split into 3 teams with distinct tasks. Only two tasks involve the \gls{cci} that concerns this paper and that is (1) generating new scenarios and question stems and (3) review and refine draft \gls{cca} questions. The second task involving creating \gls{cca} questions is out of the scope. The generation of new stems and scenarios allowed for more expansion of the questions and better incorrect options and making the \gls{cci} more well rounded. The reviewing of the \gls{cca} questions allowed for the same reviews and comments to be incorporated to the \gls{cci} either because they shared a common scenario or stem or because the comment revealed a flaw that also related to the \gls{cci}. This "hackathon" also allowed for more general feedback from experts in the field before the format of the questions was completed (SPO cite hackathon).